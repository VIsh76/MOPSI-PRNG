% \documentclass{article}
\documentclass{article}
% Packages
\usepackage[utf8]{inputenc}
\usepackage{tikz}
\usepackage{verbatim}
\usepackage{tikz}
\usetikzlibrary{calc, shapes, backgrounds}
\usepackage{amsmath, amssymb}
\usepackage{verbatim}
\usepackage{geometry}
\title{Théorie des Jeux}
\author{Victor Marchais \\ Andrei Kartashov \\ Fatine Boujnouni}
\date{Decembre 2016}


% Flipping a coin

\begin{document}
\section{Ziggurat}
\subsection*{Sur une Loi Normale, centrée réduite}
dans ce cas 
\begin{itemize}
\item La fonction est bijective de $\mathbf{R}$ dans [0, f(0)] et n'intersecte chaque rectangle qu'en un unique point.
\end{itemize}

Le calcul de V demande de calculer l'intégrale de R jusqu'à infini. (ce qu'on sait faire mais faut faire des sommes.
\end{document}